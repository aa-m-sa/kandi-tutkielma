\chapter{Johdanto}
\label{cha:johdanto}

% erillinen abstrakti palautuspdf:ään.

Tämä soveltavan matematiikan kandidaatintutkielma käsittelee matemaattisen optimoinnin menetelmää nimeltä mallinnettu jäähdytys ja sen soveltamista digitaalisen kuvankäsittelyn ja keinonäön ongelmaan: ympyräkiekkojen etsimiseen sumeasta ja kohinaisesta kuvasta.
Kuvaongelman muotoilun yhteydessä esitellään myös digitaalisen kuvankäsittelyn perusteita.

Mallinnettu jäähdytys (engl. \emph{simulated annealing}, suomeksi myös \emph{jäähdytysmenetelmä} tai \emph{simuloitu jäähdytys} \cite[suomennokset ks.][]{haataja93}, myöhemmin lyhyesti SA tai SA-algoritmi) on tunnettu probabilistinen optimointialgoritmi globaalin, tavallisesti diskreetin tai kombinatorisen optimointitehtävän ratkaisemiseksi.
(Käytettävästä terminologiasta riippuen sitä voidaan nimittää myös optimointiheurestiikaksi tai metaheurestiikaksi.)
Jäähdytysalgoritmin esittelivät \citeauthor*{kirkpatrick83} 1983 artikkelissa \cite{kirkpatrick83} Metropolis--Hastings-algoritmin sovelluksena erilaisiin kombinatorisiin optimointiongelmiin, kuten kuuluisaan kauppamatkustajan ongelmaan.
Metropolis-algoritmi vuorostaan on klassinen Markovin ketjujen Monte Carlo -menetelmä, jonka juuret ovat jäähtyvän fysikaalisen systeemin simuloinnissa.
Mallinnettu jäähdytys on yksinkertainen ja monikäyttöinen heurestiikka,
jonka hyödyllinen ominaisuus on sen kyky kiivetä ylös energiafunktion paikallisten minimien ''energiamaisemaan'' aiheuttamista kuopista~\cite{salamonetal}.

Sivuhuomiona mainittakoon että SA:lla on myös mielenkiintoisia ajankohtaisia yhtymäkohtia kvanttilaskentaan.
Metodin lähisukulainen nk. kvanttijäähdytys (\emph{quantum annealing}) perustuu siihen että monia kombinatorisia optimointiongelmia,
kuten verkon kahtiajako-ongelma, voidaan esittää spin-lasien Ising-mallin matalaenergisimpien tilojen etsintänä samaan tapaan kuin SA mallintaa termodynaamisen systeemin matalaenergisten tilojen löytämistä \cites[luku 4.5]{laarhoven}[]{johnson11dwave}.
Kvanttijäähdytysalgoritmi on väitetysti tehokas optimointimenetelmä, jos käytettävissä on sopivasti konstruoitu Ising-spin-systeemiä mallintava kvanttitietokone,
tai ainakin D-Wave Systems, Inc., esittää kykenevänsä huomattavaan laskentanopeuden parannukseen tällaisella laitteella \cite{johnson11dwave, denchev2015},
mutta nopeushypyn merkittävyydestä ei ole selvyyttä \cite{troyer2015}.

Tutkielmassa mallinnettuun jäähdytykseen tutustutaan soveltamalla sitä yksinkertaiseen kuvankäsittelyongelmaan,
jossa tehtävänä on tunnistaa tunnettu määrä $k$ tummia ympyräkiekkoja sumentuneesta ja kohinaisesta vaaleataustaisesta kuvasta.
Tutkittavaa kuvaongelmaa lähestytään optimointiongelmana:
Tutkittavista kuvista muodostetaan malli, jossa kuvan ympyräkiekot parametrisoidaan niiden sijantikoordinaattien ja säteiden avulla
ja kiekkojen sumennosta mallinnetaan kuvamatriisin konvoluutio-operaatiolla.
Minimoitavaksi \emph{kohde- eli energiafunktioksi} valitaan sopiva parametrisoitujen kiekkojen etäisyyttä kuvadatasta mittaava funktio $E$.
Energiafunktiota minimoimalla voidaan etsiä parametrit, jotka määräävät kuvadataan parhaiten sopivat kiekot.

Tutkielman pääpainopiste on mallinnetussa jäähdytyksessä ja sen käyttäytymisen tutkimisessa kuvatussa esimerkkiongelmassa.
Esitetty ympyräongelma ja valittu kuvamalli on reaalimaailman sovelluksia ajatellen varsin rajoitettu,
sillä käytännön kuvankäsittelysovelluksissa vastaavat ongelmat ovat usein monella tapaa vaikeampia:
häiriölähteitä on enemmän ja kuvan kohteista ei aina voi muodostaa yhtä yksinkertaista mallia.

Ympyräobjektien tunnistaminen kuvasta on perinteinen konenäön alan ongelma,
jota voitaisiin lähestyä myös monella muulla tapaa:
Esimerkiksi kappaleiden erottelua voitaisiin helpottaa esimerkiksi esikäsittelemällä kuvaa soveltamalla tavallisia kohinanpoistomenetelmiä tai segmentointia.
Jos sumennoksen määrä on pieni, vaihtoehtoisesti kunkin alkuperäisen kiekon keskipisteen ja säteen arviointi onnistuisi niin esikäsitellystä kuvasta jollain Houghin ympyrämuunnokseen (Circle Hough Transform) perustuvalla menetelmällä~\cite[ks esim][]{leavers92},
joille on monissa suosituissa kuvankäsittelyn kirjastoissa (esim. OpenCV, Matlab Image Processing Toolbox) valmiit toteutukset.

Mikäli kuvadata on huomattavan sumeaa, alkuperäisen sumentumattoman kuvan pinta-alan arvioiminen luotettavasti voi olla vaikeaa.
Ylipäätään sumeiden kuvien tarkentaminen (\emph{deblurring}) on myös laajalti tutkittu kuvankäsittelyn ongelma,
ja tyypillinen esimerkki epätriviaalista inversio-ongelmasta~\cite{muellersiltanen12}.
Sumennosta usein mallinnetaan konvoluutio-operaationa, jonka \emph{dekonvoluutioon} voidaan käyttää (riippuen kuinka hyvin konvoluution pistehajontafunktio tunnetaan) esimerkiksi regu\-la\-ri\-sointi- \cite{muellersiltanen12}, Wiener-suodatin-, tai tilastollisia (Bayes) menetelmiä~\cite{chaudhuri14}.
Tässä tutkielmassa sumentuneen kuvadatan ongelmaa lähestytään naiivisti olettamalla että käytössä on \emph{a priori} -informaatiota (tai muuten hyvä arvaus) kuvaa sumentaneesta prosessista, mitä hyödynnetään optimointimallissa.
(Vastakohtaisesti vaikeammassa nk. sokeassa dekonvoluutiossa tällaisia oletuksia ei tehdä,~\cite[ks. esim][]{chaudhuri14})

Tutkielman kokeellisen osuuden numeeriseen laskentaan käytettiin Matlab-ohjelmistoa~\cite{matlab2014a} ja tulosten analysointiin sekä kuvaajien tuottamiseen myös SciPy-ohjelmistoa~\cite{jones01scipy} ja sen Matplotlib-pakettia~\cite{hunter07matplotlib}.

Tutkielman rakenne on seuraava:
Luvussa~\customref{cha:simulated_annealing_algoritmi} lyhyesti todetaan joitain Markovin ketjuja koskevia esitietoja,
jonka jälkeen esitetään mallinnetun jäähdytyksen algoritmi ja sen teoreettisia perusteluja.
Luvussa~\customref{cha:kuvamalli_ja_aineisto} esitetään kiekko-ongelman matemaattinen malli ja hieman tarvittavaa kuvankäsittelyn teoriaa,
sekä malli simuloidun aineiston muodostamiseksi.
Luvussa~\customref{cha:algoritmin_soveltaminen} kuvaillaan algoritmin tarkemmin soveltamista optimointitehtävänä muotoiltuun kuvaongelmaan,
ja luvussa~\customref{cha:tulokset} esitellään tuloksia sovelletun algoritmin toiminnasta erilaisilla simuloiduilla aineistoilla.
Viimeisessä luvussa~\customref{cha:pohdinta} tarkastellaan tuloksia ja algoritmin toimintaa sekä esitetään yhteenveto tutkielman tuloksista.

%Tutkielmassa käytetyt Matlab/Octave -ohjelmakoodit ovat saatavilla GitHub-verkkopalvelussa.
