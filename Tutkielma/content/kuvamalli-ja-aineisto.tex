\chapter{Kiekko-ongelman kuvamalli}
\label{cha:kuvamalli_ja_aineisto}

\section{Digitaalinen harmaasävykuva}
\label{sec:digitaalinen_kuva}

Digitaalisessa kuvankäsittelyssä kuvan ja erityisesti valokuvan tapaisen kuvadatan tyypillinen esitysmuoto on pieninä kuvaelementteinä, \emph{pikseleinä}, (engl. \emph{pixel}, lyhenne termistä \emph{'picture element'}) ruudukossa.
Esitysmuotoa kutsutaan \emph{rasterikuvaksi} tai \emph{bittikarttakuvaksi}.
\footnote{Mainittakoon, että on myös olemassa vektorigrafiikkaa, jossa kuva esitetään geometristen objektien (kuten polkujen) avulla.
Jatkossa 'kuvalla' ja 'kuvadatalla' tarkoitetaan rasteri- / bittikarttakuvaa.}
Harmaasävykuvassa (engl. \emph{greyscale image}) kullakin pikselillä on numeerinen arvo, joka kertoo pikselin kattaman kuvapinnan alueen keskimääräisen valoisuuden eli intensiteetin.
Tästä nimi ''bittikartta'':
yleisissä tietokoneohjelmistojen käyttämissä tallennusmuodoissa pikselin valoisuus ilmaistaan 8-bittisellä kokonaisluvulla väliltä $0$ ja $2^8 -1 = 255$ (binäärinä $0_2$ ja $1111111_2$).
Vuorostaan RGB-värikuvassa pikseliin liittyy väri-informaation antava RGB-arvo (kolme erillistä 8-bittistä kanavaa punaiselle, vihreälle ja siniselle).

Harmaasävykuvan luonnollinen vastaava matemaattinen malli on $m \times n$-matriisi
\begin{equation}
    A \in [0, 1]^{m \times n},
\end{equation}
missä $m$ ja $n$ antavat kuvan koon.
Alkioiden arvot ovat reaalilukuja välillä $[0, 1]$, jotka vastaavat kyseisen pikselin valoisuutta: 0 on täysin musta pikseli, 1 valkoinen.
\cite[29--32]{jähne}

Matriisin koordinaateissa $(i,j)$ sijaitsevaa pikseliä merkitään $A(i,j)$.

Vaikka matemaattisissa malleissa kuvamatriisin alkiot käsitetään reaaliluvuiksi, tietokone laskee diskreetissä lukuavaruudessa.
%\cite[35--37]{jähne}
Tässä tutkielmassa esitellyn algoritmin käytännön Matlab-toteutuksessa laskenta tapahtuu liukulukumatriiseilla, jotka esitetään yllä kuvaillun kaltaisina pikselikuvina.
Vaikka liukulukulaskenta pystyy mallintamaan reaalilukuja vain tiettyyn tarkkuuteen asti,
ovat siitä aiheutuvat virheet niin pieniä ettei niitä tarvitse tämän tutkielman piirissä huomioida tarkemmin.

\section{Ongelmanasettelu ja kuvamalli}
\label{sec:ongelmanasettelu}

Oletetaan että tutkittavana on digitaalinen harmaasävyvalokuva, joka esittää $k$ kappaletta erillisiä mustia objekteja valkoisella taustalla,
ja objektit voidaan tulkita tasavärisiksi ympyräkiekoiksi.
Tehtävänä on määrittää kuvadatasta objektien $w_1, \dots, w_k$ määrä, sijainti ja koko.
Tutkittavien harmaasävykuvien ajatellaan syntyneen tavallisen (digitaalisen) kameran kaltaisessa systeemissä,
joten kuvamallissamme otetaan huomioon tyypillisiä tällaisessa systeemissä kuvadataa huonontavia virhelähteitä.

Matemaattinen mallimme $k$ ympyräkiekkoa $W= {w_1, ..., w_k}$ tuottavasta prosessista on
\begin{equation}
    \label{eq:kuvamalli}
    A_\text{havaittu} = P \ast A_W + E,
\end{equation}
missä havaittu lopullinen kuvadata $A_\text{havaittu}$ on ideaalinen ympyräkiekkoja esittävä kuva $A_W$, johon on kohdistunut kahdenlaisia virheitä:
kuva on sumentunut, mitä mallinnetaan konvoluutio-operaatiolla $P\ast A_W$, ja siinä on additiivista kohinaa $E$.

Seuraavaksi määritellään ideaalisen kuvamatriisin $A_W$ tuottaminen kiekkokokoelmasta $W$, konvoluutio-operaatio ja additiivinen kohina.
Kuvaillaan myös kuinka tätä yleistä kuvamallia sovelletaan simuloidun aineiston tuottamiseen ja aineiston tutkimiseen käytetty 'havaintomalli',
jota hyödynnetään luvussa~\ref{cha:algoritmin_soveltaminen} kiekkojen löytämiseen simuloidusta aineistosta.

\section{Ympyräkiekot}
\label{sec:kiekot}

Tavoitteena on rekonstruoida kuvadatan pohjalta ympyräobjektit, joista kukin voidaan yksilöidä järjestettynä kolmikkona (tai vektorina)
\begin{equation}
    w = (x_0, y_0, r) \in m\times n\times \R,
\end{equation}
missä $x_0, y_0$ on ympyrän keskipisteen koordinaatti $m \times n$ kuvamatriisissa $A$ ja $r$ ympyrän säde.

\subsection{Yksi kiekko}
\label{sub:yksi_kiekko}

\emph{A priori} -tietomme kuvan esittämien kappaleiden muodosta ja luonteesta mahdollistaa niiden oleellisen informaatiosisällön esittämisen kolmella parametrilla. Tulkitaan että kolmikko $(x_0, y_0, r)$ määrää yhden mustan ympyrän valkoisella pohjalla kuvamatriisissa $A$ seuraavan funktion avulla:
\begin{maar}
    Ympyrä, jonka keskipiste on $(x_0, y_0)$ ja säde $r > 0$, voidaan kuvata kuvamatriisiksi $A$ funktiolla
    $f: m\times n\times \R \to [0, 1]^{n\times n},$
    \begin{equation*}
        f(x_0, y_0, r) = A, \text{missä }
        \begin{cases}
            A_{i,j} &= 0, \text{ jos pikselin $(i,j)$ keskipiste on ympyrän sisällä}\\
            A_{i,j} &= 1, \text{ muulloin}
        \end{cases}
    \end{equation*}
    missä ympyrän (ympyräkiekon sisällä) jos $\norm{(i, j) - (x_0, y_0)} \leq r$.
\end{maar}

Teimme yllä muutaman tietoisen yksinkertaistuksen numeerisen laskennan helpottamiseksi.
Ensinnäkin oletamme että ympyräkiekkojen keskipisteet sijaitsevat aina tarkalleen jonkin pikselin kohdalla ($x_0, y_0$ kokonaislukuja jotka vastaavat pikselien koordinaatteja);
realistisemmin olettaisimme koordinaatit $x_0$ ja $y_0$ reaaliluvuiksi kuvamatriisin alueella, ja laskisimme mitkä pisteet kuuluvat ympyräkiekkoon laskemalla niiden etäisyydet niihin.
Toiseksi mallimme ympyräkiekosta on varsin karkea graafisessa mielessä:
pikseli on joko täysin musta tai valkea.

\subsection{Monta kiekkoa}
\label{sub:monta_kiekkoa}

Malli voidaan yleistää useammalle kiekolle.
Oletetaan että meillä on $k$ kappaleen (järjestämätön) kokoelma kiekkoja $W = \left\{w_1, \dots, w_k\right\}$.
Intuitiivisesti vastaavassa kuvamatriisissa pikseli on valkea, jos se ei kuulu mihinkään ympyräkiekkoon, ja musta jos se kuuluu johonkin kiekkoon.
Kokoelma on järjestämätön, sillä tulokseksi saadaan samanlainen kuva riippumatta missä järjestyksessä sen ympyrät luetellaan.

\begin{maar}
    $k$ ympyräkiekkoa $w_1, \dots, w_k$ voidaan kuvata kuvamatriisiksi $A$ funktiolla $f_{\text{monta}} : \left( m \times n \times \R \right)^k \to [0, 1]^{m \times n},$
    \begin{equation*}
        f_{\text{monta}}(w_1, \dots, w_k) = A, \text{missä }
        %\begin{cases}
        \left\{
        \begin{aligned}
            A_{i,j} = 0, &\text{ jos pikselin $(i,j)$ keskipiste on}\\
             & \text{ jonkin ympyrän $w_i$ sisällä}
            \\
            A_{i,j} = 1, &\text{ muulloin}
        \end{aligned}
        \right.
        %\end{cases}
    \end{equation*}
\end{maar}
Suoraan nähdään että kokoelman $W$ vektoreiden $w_1, \dots, w_k$ järjestyksellä funktion $f_{\text{monta}}$ parametreina ei ole väliä,
\begin{equation}
    f_{\text{monta}}(w_1, \dots, w_i, w_j, \dots, w_k) = f_{\text{monta}}(w_1, \dots, w_j, w_i, \dots, w_k), \quad i \not = j.
\end{equation}
Näin ollen merkintä $f_{\text{monta}}(W)$ on järkevä.

Näin määritellyn kiekkomallin avulla voidaan tulkita ympyräkokoelma $W$ vastaavaksi kuvamatriisiksi $A = A_W = f_{\text{monta}}(W)$,
ja jatkossa teemme näin suoraan mainitsematta funktiota $f_{\text{monta}}$ eksplisiittisesti.

Funktio on helppo toteuttaa ohjelmallisesti ''ja'' -operaatiolla $\wedge$ matriiseille:
\begin{maar}
Jos $A = B \wedge C$, missä $A, B, C$ ovat $m \times n$ -matriiseja, niin
\begin{equation}
    A_{i,j} =
    \begin{cases}
        &1, \text{ jos } B_{i,j} = 1 \text{ ja } C_{i,j} = 1 \\
        &0, \text{ jos } B_{i,j} = 0 \text{ tai } C_{i,j} = 0.
    \end{cases}
\end{equation}
\end{maar}

Jos $A_i$ on ympyräkiekkoa $w_i$ vastaava kuvamatriisi eli $A_i = f(w_i), i = 1, \dots, k$, voimme käsitellä kuvamatriiseja $A_i$ \emph{maskeina} ja kirjoittaa koko kiekkokokoelmaa vastaavan kuvamatriisin $A_W$ niiden avulla
    \begin{equation}
        A_W = A_1 \wedge \dots \wedge A_k.
    \end{equation}

\section{Konvoluutio}
\label{sec:konvoluutio}

Digitaalisessa kuvankäsittelyssä kuvan sumentaminen (engl. \emph{blurring}, tai pehmentäminen, engl. \emph{smoothing}) tehdään soveltamalla kuvamatriisiin $A$ sopivaa (lineaarista) suodatinta $P$ (engl. \emph{filter}).
Tuloksena saadaan summennettu kuva $A_\text{sumea}$, jossa kunkin pikselin $A_\text{sumea}(i,j)$ arvo perustuu vastaavan alkuperäisen, terävän kuvamatriisin $A$ pikselin $A(i,j)$:n ja sen ympäristön arvoihin.
Tämä on intuitiivinen analogia fysikaaliselle ilmiölle, jossa virheellisesti tarkennetun optisen systeemin linssi- tai peilijärjestelmä kohdistaa havaittavasta kohteesta peräisin olevan valon epätarkasti kameran CCD-kennon tai silmän verkkokalvon kaltaiselle havaintopinnalle ja tuottaa sumean kuvan.

Käytännössä $A_\text{sumea}(i,j)$ saadaan ottamalla $A(i,j)$:n ympäristön pikseleistä painotettu keskiarvo.
Painotetun keskiarvon painot ja sen, mitä oikeastaan tarkoitetaan 'ympäristöllä' kertoo suodatinmatriisi $P$.
Formaalisti tällainen suodatus vastaa $A$:n ja $P$:n (lineaarista) \emph{konvoluutiota} $P \ast A$, jonka määrittelemme seuraavaksi.
Sanomme joskus lyhyesti että kuvamatriisille $A$ tehdään (lineaarinen) konvoluutio-operaatio $P \ast$.
\cite{burger09}

Yleisesti funktioteoriassa konvoluutio on kahden funktion $f$ ja $g$ välinen operaatio, joka määrittelee uuden funktion $f \ast g$.
\begin{maar}
    Funktioiden $f$ ja $g$ \emph{konvoluutio} $f \ast g$ on
    \begin{align*}
        f \ast g = (f \ast g) (x)
        &= \int_{-\infty}^{\infty}f(t)g(x - t)\df{t}
    \end{align*}
\end{maar}

\begin{lause}
    Konvoluutio on vaihdannainen, $f \ast g = g \ast f$:
\end{lause}

\begin{proof}
    Todistus seuraa soveltamalla muuttujanvaihtokaavaa epäoleelliselle integraalille:
    \begin{align*}
        f \ast g = (f \ast g) (x)
        &= \int_{-\infty}^{\infty}f(t)g(x - t)\df{t} \\
        &= \lim_{a \to \infty} \int_{-a}^a f(t)g(x-t)\df{t}, \\
        \intertext{tehdään muuttujanvaihto $t = x - s, \df{t} = -\df{s}$, jolloin}
        &= \lim_{a \to \infty} \int_{x + a}^{x - a} -f(x - s)g(s)\df{s}\\
        &= \lim_{a \to \infty} \int_{x - a}^{x + a} f(x - s)g(s)\df{s}\\
        &= \int_{-\infty}^{\infty}g(s)f(x -s)\df{s},
    \end{align*}
    missä viimeinen integraali on haluttua muotoa ja määritelmän mukaan $g \ast f$.
\end{proof}

Mikäli funktioiden $f, g$ määrittelyjoukko on kokonaislukujen joukko, integraali voidaan luonnollisella tavalla korvata summana.
Toisin sanoen voidaan määritellä \emph{diskreetti konvoluutio}:

\begin{maar}
    Diskreetti konvoluutio on
    \[
        (f \ast g) (n) = \sum^{\infty}_{m=-\infty} f(m) g(n - m),
    \]
    joka yleistettynä kahteen ulottuvuuteen $\Integer^2$ on
    \[
        (f \ast g)(n_1, n_2) =
        \sum^{\infty}_{m_1=-\infty} \sum^{\infty}_{m_2=-\infty}
        f(m_1, m_2) g(n_1-m_1, n_2-m_2).
    \]
\end{maar}

Funktioille määritelty konvoluutio voidaan helposti laajentaa matriiseille samastamalla yleinen reaalinen matriisi $B \in \R^{m \times n}$ sen määräämän funktion $f_B:~m \times n \to R^{m \times n}$ kanssa, missä $f_B$ saa arvonsa matriisista $B$
\begin{equation}
    f_B(i, j) = b_{i,j} = B(i,j),
\end{equation}
missä $b_{i,j}$ on $B$:n alkio $i.$ rivillä ja $j.$ sarakkeessa.

Toisin sanoen merkintämme $A$:n konvolvoimiselle $P$:llä
\begin{equation}
    A_\text{sumea} = P \ast A
\end{equation}
on järkevä.
Määritellään vielä konvoluutiomatriisi $P$ tarkemmin ja tutkitaan kuinka sumennos käytännössä konvoluutiossa tapahtuu.

\begin{maar}
    Konvoluution $P \ast$ ydin (engl. \emph{kernel}) eli konvoluutiomatriisi eli suodatinmatriisi $P$ on äärellinen $K \times L$ -matriisi joillekin luonnolliselle luvulle $K, L \in \Natural$.
    Konvoluutio-operaatiossa $P$:n ajatellaan olevan 0 $K \times L$ -matriisin ulkopuolella.
\end{maar}

Oletetaan että $P$ on $3 \times 3$ konvoluutiomatriisi, joka on indeksoitu niin että keskimmäinen alkio on $p_{0,0}$, ja $A$ on $m \times n$ -kuvamatriisi.
Tällöin diskreetti kaksiulotteinen konvoluutio $P \ast A$ saadaan määritelmän mukaan seuraavasta yhtälöstä, kun $1 \leq k \leq m$ ja $1 \leq l \leq n$:
\begin{align}
    \label{eq:konvoluutiosumma}
    (P \ast A)(k,l) &= \sum^{\infty}_{i=-\infty} \sum^{\infty}_{j=-\infty} P(i, j) A(k - i, l - j) \\
    \intertext{Ottaen huomioon että $P$:n arvo konvoluutiomatriisin ulkopuolella on 0 eli $3 \times 3$ matriisin tapauksessa $P(i, j) = 0$ kun $ \abs{i} > 1$ tai $\abs{j} >1$, voidaan kirjoittaa:}
        &=  \sum_{i = -1}^1\sum_{j = -1}^1 P(i, j) A(k - i, l - j) \\
        &=  \mathop{\sum\sum}_{i, j = -1, 0, 1} P(i, j) A(k - i, l - j)
\end{align}

Toisin sanoen $P \ast A$ voidaan tulkita $n \times n$ -kuvamatriisiksi, jonka kunkin $(k,l)$-pikselin $(P \ast A)(k,l)$ arvo on painotettu summa $A(k,l)$:sta ja sen naapuripikseleistä, missä painot saadaan konvoluutiomatriisista.

\subsection{Reunat}
\label{sub:reunat}

Vielä on sovittava kuinka käsitellään $A$:n reunat,
eli tapaukset jolloin kaavassa~\ref{eq:konvoluutiosumma} summattava $A(i, j)$ ei ole määritelty koska indeksit $(i, j)$ menevät $A$:n rajojen ulkopuolelle, $i < 1$ tai $i > m$ tai $j < 1$ tai $j > n$.

Yksinkertaisin vaihtoehto olisi yksinkertaisesti jättää summasta~\ref{eq:konvoluutiosumma} pois ne pikselit joissa konvoluutio-operaatio ei näin tulkittuna olisi määritelty;
tällöin konvolvoitu matriisi $P \ast A$ olisi mitoiltaan pienempi kuin alkuperäinen $A$, mikä ei ole toivottavaa.

Toinen usein käytetty vaihtoehto on 'laajentaa matriisia nollalla', ja tulkita matriisin $A$:n pikseleiden $A(i,j)$ arvot nollaksi kun $i,j$ ovat matriisin ulkopuolella.
Koska valitussa kuvamallissa kuvamatriisin $A$:n arvot 'neutraalia' tyhjää taustaa esittävillä alueilla ovat $1$ ja kappaleen kohdalla $0$ (kuvamme esittää tummia kappaleita valkealla taustalla), käytetään 'nollana' lukua 1.
\footnote{Tai vaihtoehtoisesti tuottaa kuvadatasta käänteiskuva ja operoida sillä: matriisin alkio on $1 = \text{kuuluu ympyräkiekkoon}, 0 = \text{ei kuulu kiekkoon}$.}
Ongelmaksi jää tällöin tilanne, jossa ympyräkiekko ei ole kokonaan kuvan sisällä: tällaiset kiekot saisivat kuvan laidassa valkean reunan.

Kolmas yleinen ratkaisu on laajentaa kuvamatriisia konvoluutiota laskettaessa toistamalla 'nollan' sijasta $A$:n olemassaolevia reunapikselejä.
Koska puuttuvien konvoluution laskemiseen tarvittavat alueet ekstrapoloidaan, tämäkin aiheuttaa pienen vääristymän.
\footnote{Parhaassa tapauksessa konvolvoitava raakakuva olisi suurempi kuin tarpeen, jolloin siitä voitaisiin leikata reunat pois ja syntyvä pienempi kuva olisi toivottun kokoinen.}
Tutkielmassa käytetyillä pienillä konvoluutiomatriiseilla vääristymä on kuitenkin hyvin pieni.
\cite{burger09}

Joissain tapauksissa on hyödyllistä tulkita kuva \emph{toruspintana},
jolloin kunkin reunan yli menevät pikselit tulkitaan otettavaksi vastakkaisesta reunasta.
Näin määriteltyä konvoluutiosuodatinta sanotaan myös \emph{sykliseksi} konvoluutioksi.
Käytännössä ero siihen että kuvamatriisia jatkettaisiin on myös pienillä konvoluutiomatriiseilla pieni.
\cite[104--106]{jähne}

Vast'edes tutkielmassa sovitaan, että konvoluutio-operaatiossa reunat käsitellään kolmannella esitetyllä tavalla eli jatkamalla olemassaolevia reunapikseleitä.

\section{Kohina}
\label{sec:kohina}

Tavoitteena on mallintaa prosessia, jonka tuottamat kuvat sumentuneen lisäksi kohinaisia, mihin käytämme additiivista kohinamallia.

Yleisesti kohinalla tarkoitetaan havaitussa kuvassa $A_\text{havaittu}$ ei-toivottua satunnaista komponenttia.
Additiivisessa kohinamallissa tätä mallinnetaan lisäämällä kohinattomaan kuvaan satunnaismatriisi $E$,
jossa jokainen pikseli alkio (pikseli) on jokin satunnaismuuttuja.
Jos sumennettu (konvolvoitu) kuva on $P \ast A$, lopullinen sumentunut ja kohinainen kuva on
\begin{equation*}
    P \ast A + E.
\end{equation*}
Tällaisessa mallissa kohinamatriisin $E$ jakauma tyypillisesti riippuu mallinnettavasta häiriölähteestä.
Monia yleisiä kohinatyyppejä mallinnetaan normaalijakautuneella kohinamallilla (\emph{Gaussian noise}),
kuten esimerkiksi laitteiston elektronisten komponenttien lämpökohinaa (\cite{boncelet05bovik}) tai approksimaationa Poisson-jakautuneesta otoskohinasta \cite[Ks. esim.][ luku 3.4.5]{jähne}.
Toinen yleinen kohinatyyppi on nk. suola ja pippuri -kohina (\emph{salt and pepper noise}), jota syntyy esimerkiksi digitoinnissa tai digitaalisessa tiedonsiirrossa: kuvan pikseleistä vain harvaan on kohdistunut virhe, mutta häiriöt ovat hyvin suuria suuntaan tai toiseen,
jolloin kuva näyttää siltä ikään kuin sille olisi ripoteltu suolaa ja pippuria \cite{boncelet05bovik}.

Yhteistä edellämainituille virhelähteille on, että kohina syntyy vasta kuvan digitaalisessa talteenottolaitteistossa:
kuva on jo sumea havaintopinnan tulkitessa sen digitaaliseksi signaaliksi.
Tässä mielessä additiivinen malli on mielekäs.


\section{Havaintomalli ja simuloidun aineiston malli}
\label{sec:simuloitu_aineisto}

Edellä määriteltiin yleinen kuvamalli
\begin{equation*}
    \tag{\ref{eq:kuvamalli}}
    A_\text{havaittu} = P \ast A_W + E
\end{equation*}
sumentuneita ja kohinaisia kiekkokuvia tuottavasta prosessista.

Tutkielmassa esitellyn algoritmin toimivuutta esitettyyn ongelmaan testataan kokeilemalla sitä simuloituun aineistoon.
Oikeasssa tutkimustilanteessa havaintomalli olisi muodostettu approksimoimaan jotain todellista prosessia,
ja fysikaalisesta prosessista muodostettu malli on usein ainakin hieman epätarkka, sillä tutkittava prosessi harvoin tunnetaan aivan täydellisesti.

Näin ollen olisi epärealistista sekä tuottaa simuloitu aineisto että sitten analysoida sitä täsmälleen samanlaiseen malliin perustuen.
Oletetaan kuitenkin että havaintomallimme on approksimoi todellisuutta varsin hyvin,
jolloin simuloituun aineistoon $A_{\text{data}}$ voidaan käyttää periaatteeltaan samaa mallia
\begin{equation}
    \label{eq:aineistomalli}
    A_{\text{data}} = P_{\text{data}} \ast A_W + E,
\end{equation}
missä datan luomiseen käytetty sumentava suodatin on kuitenkin hieman erilainen kuin havaintomallissa odotetaan.

Kuvittelememme siis että tutkimme yleisen mallin~\ref{eq:kuvamalli} mukaista prosessia,
jonka ''todelliset parametrit'' (konvoluutiomatriisi) ovat mallin \ref{eq:aineistomalli} mukaiset.
Ympyröiden tunnistamiseen (seuraavasssa luvussa) kuitenkin käytetään nk. ''havaintomallia'',
joka on hieman yksinkertaisempi kuin aineiston luomiseen käytetty ''todellinen'' malli.

Käytännössä ero mallien välillä on pieni, mutta tarkoituksena on esitellä parhaita periaatteita kuinka tällaisissa tutkimusongelmissa ''realistinen'' simuloitu aineisto kannattaa tuottaa.
Erityisesti monissa inversio-ongelmien laskennallisissa ratkaisuissa aineiston tuottaminen ja inversioratkaisun laskeminen samalla laskentamenetelmällä voi johtaa epärealistisen hyviin tuloksiin~\cite{muellersiltanen12}.

\subsection{Havaintomallin konvoluutiomatriisi}
\label{sub:mallin_konvoluutiomatriisi}

Valitaan havaintomallissa käytettäväksi konvoluutiomatriisiksi $P$
\begin{equation}
    P = \frac{1}{26} \cdot
    \begin{pmatrix}
        2 & 2 & 2 \\
        2 & 10 & 2 \\
        2 & 2 & 2
    \end{pmatrix}.
\end{equation}
Yllä matriisi kerrotaan skalaarilla $\frac{1}{26}$, jolla \emph{normitetaan} matriisi siten että konvoluutio vastaisi \emph{painotetun keskiarvon} ottamista.
Skalaari $1/26$ saadaan yhtälöstä \[\frac{1}{26}(8\cdot2 + 10) = 1.\]

Valittu $P$ muistuttaa hieman kuvan- ja signaalinkäsittelyssä käytettyä Gauss-sumennosta (engl. \emph{Gaussian low pass filter, Gaussian blur}), jossa konvoluutiomatriisin arvot saadaan kaksiulotteisesta Gaussin normaalijakaumasta.

\subsection{Havaintomallin kohina}
\label{sub:kohinamalli}

Havaintomallissamme oletetaan yksinkertaisesti että kohina $E$ on tasaisesti jakautunutta valkoista kohinaa.
Käytettävä simuloitu data (joka määritellään tarkemmin alaluvussa~\ref{sec:simuloitu_aineisto}) ei pyri mallintamaan mitään tiettyä erityistä kamerasysteemiä,
joten voimme valita yleisen esimerkin kohinalähteestä joka huonontaa kuvan laatua varsin paljon.
Oikeassa sovelluksessa kohinamallia todennäköisesti olisi tarpeen tarkentaa käytettävän mittaustilanteen ja -välineiden oikeasti tuottaman kohinan perusteella.

Kohinamatriisi $E$ ajatellaan satunnaismatriisiksi, jonka alkiot $E(i,j) = E_{i,j}$ ovat riippumattomia tasaisesti jakautuneita satunnaismuuttujia keskiarvolla 0,
\begin{equation}
    E(i,j) \sim U(-0.5, 0.5)
\end{equation}
Tällaista kohinaa, jossa $E_{i,j}$ ovat samoin jakautuneita (riippumattomia) satunnaismuuttujia ja niiden keskiarvo on 0, kutsutaan valkoiseksi kohinaksi.\footnote{Valkoisen kohinan tarkka määritelmä joka perustuu $E$:n tehospektrin  tarkasteluun ei kuulu tämän tutkielman piiriin. Tehospektri voidaan määritellä esimerkiksi $E$:n autokorrelaatiofunktion Fourier-muunnoksen kautta, \cite[ks.][s.96 -- 97]{jähne}.}

Visuaalisesti summamatriisin $P \ast A + E$ alkiot joiden arvot ovat välin $[0, 1]$ ulkopuolella tulkitaan luvuksi $0$ tai $1$ riippuen olisiko alkio välin $[0, 1]$ ala- vai yläpuolella.

\subsection{Simuloidun aineiston sumennos ja kohina}
\label{sub:simuloidun_aineiston_sumennos_ja_kohina}

Sumentavaksi suodattimeksi $P_{\text{data}}$ valitaan hieman monimutkaisempi konvoluutiomatriisi, jota havaintomallin konvoluutiomatriisi yrittää ''epätarkasti'' mallintaa,
\begin{equation}
    P_{\text{data}} =
    \frac{1}{36}
    \begin{pmatrix}
        0 & 1 & 1 & 1 & 0 \\
        1 & 2 & 2 & 2 & 1 \\
        1 & 2 & 8 & 2 & 1 \\
        1 & 2 & 2 & 2 & 1 \\
        0 & 1 & 1 & 1 & 0 \\
    \end{pmatrix}.
\end{equation}

Simuloituun aineistoon lisättävä kohina on havaintomallin mukaista tasaista satunnaiskohinaa
\begin{equation}
    E(i,j) \sim U(-0.5, 0.5)
\end{equation}
jota on helppo generoida laskennellisilla ohjelmistoilla.
Havaintomallin mukaisesti lopullisessa kuvassa harmaasävykuvan pikseleiden arvorajojen $[0, 1]$ ylle tai alle arvoja saavat matriisin alkiot tulkitaan pikseleiksi
\begin{equation}
    \label{eq:leikkaus}
    (P \ast A + E)_\text{kuva} (i,j) =
    \begin{cases}
        0 , &\text{jos } (P \ast A + E)(i,j) < 0\\
        1 , &\text{jos } (P \ast A + E)(i,j) > 1.
    \end{cases}
\end{equation}
